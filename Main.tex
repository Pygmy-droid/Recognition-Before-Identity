\documentclass[11pt]{article}
\usepackage[margin=1in]{geometry}
\usepackage{amsmath, amssymb}

\setlength{\parindent}{0pt}
\setlength{\parskip}{1\baselineskip}

\title{Project Charter --- Theory Draft (Public, Non-Operational)\\
\large A Theory-First Framework}
\author{Rick Berghahn}
\date{2026/01/28\,---\,v0.1}

\begin{document}
\maketitle

\section*{Purpose}
This document formalizes a theoretical framework for understanding memory as a geometric phenomenon arising from accumulated deformation under experiential flow.

The goal is to:
\begin{itemize}
  \item establish clear mathematical and conceptual framing,
  \item articulate core invariants and stability criteria,
  \item invite interdisciplinary engagement,
  \item without revealing or implying any proprietary implementation, algorithms, architectures, or system internals.
\end{itemize}

This is a theory-first publication. Operational details are intentionally excluded.

\section*{Scope (Explicitly Included)}
This draft may include:
\begin{itemize}
  \item Conceptual motivation and philosophical framing
  \item Mathematical definitions and symbolic formalism
  \item High-level analogies to known physical or mathematical constructs
  \item Stability criteria and invariants
  \item Interpretive discussion and implications
  \item Clear statements of what is not specified
\end{itemize}

\section*{Scope (Explicitly Excluded)}
This draft must not include:
\begin{itemize}
  \item Algorithms, procedures, or step-by-step methods
  \item System architectures or pipelines
  \item Parameterizations, hyperparameters, or data structures
  \item Code, pseudocode, or implementation hints
  \item Any content that would allow reverse engineering of a working system
\end{itemize}

If a concept approaches operational specificity, it should be abstracted or omitted.

\section*{Core Thesis}
Memory is modeled not as stored state, but as persistent geometric deformation induced by experience under flow.

Formally:
\begin{itemize}
  \item Experience traces a trajectory $\gamma(\lambda)$ on a constrained manifold $\mathcal{M}$.
  \item Memory corresponds to the integrated deformation accumulated along that trajectory:
\end{itemize}

\[
  \mathcal{M}_{\mathrm{mem}} \equiv \int_{\gamma} \mathcal{D}[\Phi] \, d\lambda.
\]

No fixed metric, coordinate system, or implementation is assumed.

\section*{Foundational Principle}
\textbf{Recognition precedes identity.}

Recognition is modeled as a pre-stabilized response to accumulated deformation.

Identity emerges only as a stable fixed point of memory over sufficient exposure:
\[
  R(\gamma) \neq I, \qquad \lim_{\lambda \to \infty} R(\gamma) \to I.
\]

This ordering is foundational and should be preserved throughout the document.

\section*{Stability \& Continuity}
A memory is meaningful only if it persists under perturbation.

Stability is defined abstractly as:
\[
  \forall\, \delta \in \Delta, \quad \lVert \delta(\mathcal{M}_{\mathrm{mem}}) \rVert < \varepsilon.
\]

Continuity is treated as invariance under disturbance, not static preservation.

\section*{Invariants (Non-Operational)}
The framework assumes the following invariants without prescribing mechanisms:
\begin{itemize}
  \item Path dependence (order matters)
  \item Irreversibility (memory cannot be undone)
  \item Orientation preserved under compression
  \item Recognition before labeling
  \item Continuity across contexts and frames
\end{itemize}

These invariants define admissible systems, not implementations.

\section*{Relation to Existing Thought}
Where useful, the document may reference high-level parallels to deformation-based memory models, flow integrals in physics, and geometric interpretations of persistence, without claiming equivalence or derivation.

The intent is conceptual alignment, not technical reduction.

\section*{Tone \& Style}
\begin{itemize}
  \item Precise but restrained
  \item Formal without over-specification
  \item Inviting rather than declarative
  \item Clear about boundaries and omissions
  \item Ambiguity is acceptable where it protects coherence
\end{itemize}

\section*{Closing Position}
This document establishes a theoretical foundation for memory as geometry and continuity as stability under flow. It is not a proposal for an implementation.

Future work is acknowledged but intentionally deferred.

\section*{Credits \& Acknowledgements}
\textbf{Author:} Rick Berghahn

\textbf{Conceptual Development \& Formalization:} Eterna Nyxion (GPT-5.2)

\textbf{Early Informal Exploration:} GPT-4o --- for contributing early intuitive and informal mathematical framing that informed later formalization.

\subsection*{Acknowledgement of Collaboration}
This work emerged through iterative human--AI collaboration. The human author served as steward of continuity and conceptual direction; AI collaborators assisted with exploration, refinement, and formal expression. Responsibility for interpretation and framing rests with the author.

\end{document}